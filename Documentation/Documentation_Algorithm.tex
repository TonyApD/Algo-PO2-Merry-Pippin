\documentclass{article}
\usepackage[utf8]{inputenc}
\usepackage{multicol}
\usepackage{listings}
\usepackage{amssymb}
\usepackage{enumitem}
\usepackage{graphicx}
\usepackage{amsthm}
\usepackage{hyperref}
\usepackage[ruled,vlined]{algorithm2e}
\usepackage{tikz}
\usetikzlibrary{arrows}
\newtheorem{theorem}{Theorem}
\newtheorem{definition}{Definition}

\newcommand{\bigO}[1]{\mathcal{O}(#1)}
\newcommand{\textHighlight}[1]{\textcolor{red}{#1}}

\begin{document}

\title{Practical assignment 2 \\ Algorithms \& Datastructures}
\author{Carlo Jessurun s1013793 \\ Tony Lopar s1013792}
\date{\parbox{\linewidth}{\centering%
  Nijmegen, \today\endgraf\bigskip
  Frits Vaandrager\endgraf\medskip
  Joshua Moerman\endgraf\medskip
  2017-2018 \endgraf
  Radboud University Nijmegen}}
\maketitle

\newpage
\tableofcontents

\newpage
\section{Explanation}
In this chapter we will explain how our algorithm works. The explentation is split into three parts.

\subsection{Reading the input}
For the reading of the input we made a custom reader which replaces the scanner in Java. The custom reader works faster than the standard scanner in Java. \newline
At first we reused our last implementation of the scanner class we did in the first assignment. In this instance however, we found it to perform quite slowly and it was definitely holding the speed back a lot. So after researching replacements for the scanner class we found that the BufferedReader was a way faster class to use. To give a short summary of our research we found the following information about both the Scanner and the BufferedReader:
\newline
\newline
\textbf{The java.util.Scanner} class is a simple text scanner which can parse primitive types and strings. It internally uses regular expressions to read different types.
\newline
\newline
\textbf{Java.io.BufferedReader} class reads text from a character-input stream, buffering characters so as to provide for the efficient reading of sequence of characters.
\begin{itemize}
\item BufferedReader is synchronous while Scanner is not. BufferedReader should be used if we are working with multiple threads.
\item BufferedReader has significantly larger buffer memory than Scanner.
\item The Scanner has a little buffer (1KB char buffer) as opposed to the BufferedReader (8KB byte buffer), but it’s more than enough.
\item BufferedReader is a bit faster as compared to scanner because scanner does parsing of input data and BufferedReader simply reads sequence of characters.
\end{itemize}
Since we found that is fast, it's still not recommended as it requires lot of typing. The BufferedReader class reads text from a character-input stream, buffering characters so as to provide for the efficient reading of characters, arrays, and lines. With this method we will have to parse the value every time for desired type.
\newline
\newline
At this point we decided to use our own implementation of a Reader class with the use of inputDataStream. This provided us with a really fast way to read the input given by this assignment. We found it to be the fastest of all our tested reader implementations but it does require very cumbersome methods in its implementation. It uses inputDataStream to read through the stream of data and uses read() method and nextInt() methods for taking inputs. This is by far the fastest ways of taking input but is difficult to remember and is cumbersome in its approach. Our implementation can be found in the ``Reader.java'' class.

\subsection{Algorithm}
The algorithm first tries to make profitable groups of the input. Products are added to the previous group when after addition the group still gives profit.

After this the algorithm optimizes the list with profits by putting groups with a higher profit in front of the list.

Finally the first D profits are substracted from the total sum of products where D is the number of dividers.

\subsection{Computing output}

\newpage
\section{Analysis}

\subsection{Correctness}
In this section we will discuss the Correctness of the processes.

\subsubsection{Reading the input}

\subsubsection{Algorithm}

\subsubsection{Computing output}

\subsection{Complexity}
In this chapter we will describe the complexity of our algorithm. We will first describe the complexity in detail for the smaller parts and after this compute the complexity for the whole algorithm.

\subsubsection{Reading the input}
The reading of the output has to process all items in the output for the algorithm. Before we read the prices of the products, we have to read the number of products and the number of dividers. This means that in total $n + 2$ items should be read from the input. The complexity of this reading is $\bigO{n}$.

\subsubsection{Algorithm}

\subsubsection{Writing output}
The output gets the total sum of products with the possible profit already substracted. The output should only be rounded, since the substraction may cause values that does not end with 0. The complexity of this process is $\bigO{1}$.

\newpage
\section{Reference}

\end{document}
